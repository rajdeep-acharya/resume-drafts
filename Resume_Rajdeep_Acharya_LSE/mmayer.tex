%%%%%%%%%%%%%%%%%%%%%%%%%%%%%%%%%%%%%%%%%
%%% Resume Rajdeep Acharya 06/2021 %%%%%%
%%%%%%%%%%%%%%%%%%%%%%%%%%%%%%%%%%%%%%%%%

\PassOptionsToPackage{dvipsnames}{xcolor}
\documentclass[9.5pt,a4paper]{altacv}

%Layout
\geometry{
left=0.5cm,
right=9cm,
marginparwidth=8.0cm,
marginparsep=0.5cm,
top=0.5cm,
bottom=0.5cm,
footskip=1
\baselineskip}

%Packages
\usepackage[utf8]{inputenc}
\usepackage[T1]{fontenc}
\usepackage[default]{lato}
% \usepackage[sfdefault]{carlito}
% \usepackage[default]{tgbonum}
\usepackage{hyperref}
% \usepackage{ragged2e}

%Colors
\definecolor{accent}{HTML}{000e17}
\definecolor{heading}{HTML}{000e17}
\definecolor{emphasis}{HTML}{696969}
\definecolor{body}{HTML}{01415f}
% \definecolor{body}{HTML}{008080}
\colorlet{heading}{heading}
\colorlet{accent}{accent}
\colorlet{emphasis}{emphasis}
\colorlet{body}{body}

\renewcommand{\itemmarker}{{\small\textbullet}}
\renewcommand{\ratingmarker}{\faCircle}

%

\begin{document}
% \justifying
\name{Rajdeep Acharya}
\tagline{Lead Software Engineer}

% Cropped to square from 
% https://en.wikipedia.org/wiki/Marissa_Mayer#/media/File:Marissa_Mayer_May_2014_(cropped).jpg, 
% CC-BY 2.0
% \photo{3.3cm}{profile.jpg}
% \photoR{2.5cm}{Rajdeep_photo}

\personalinfo{
    % Not all of these are required!
    % You can add your own with \printinfo{symbol}{detail}
    % \photo{https://drive.google.com/file/d/1xh8rK3gGejIKz3cdHLoS5sQhKqFepNuI/view}{1}
    \email{rajdeep777acharya@gmail.com}
    % \hspace{1em}
    \phone{+91-9663970562}
%   \mailaddress{}
%   \homepage{}
%   \twitter{}
    % \newline
    \linkedin{rajdeep-acharya/}{linkedin.com/in/rajdeep-acharya/}
    \github{rajdeep-acharya}{github.com/rajdeep-acharya}
    % \newline
    \location{Bangalore, India}
    % If you want to use this field (and also other academicons symbols), 
    % add "academicons" option to \documentclass{altacv}
}


% https://avatars.githubusercontent.com/u/23383519?s=400&u=9474bfbb96040db071c84fd5b27fe59045bfc07d&v=4
% https://drive.google.com/file/d/1xh8rK3gGejIKz3cdHLoS5sQhKqFepNuI/view


%%%%%%%%%%%%%%%%%%%%%%%%%%%%%%% Professional Summary %%%%%%%%%%%%%%%%%%%%%
\cvsectionComplete{Professional Summary}
{\textbf{9.5 years of experience} in software design, development, integration, maintenance of various client-server web applications.}

%% Make the header extend all the way to the right, if you want.

\begin{fullwidth}
\makecvheader
\end{fullwidth}

%% Depending on your tastes, you may want to make fonts of itemize environments slightly smaller

\AtBeginEnvironment{itemize}{\small}



%%%%%%%%%%%%%%%%%%%%%%%%%%%%%%% Experience %%%%%%%%%%%%%%%%%%%%%%%%%%%%%%%

%% Provide the file name containing the sidebar contents as an 
%% optional parameter to \cvsection.
%% You can always just use \marginpar{...} if you do
%% not need to align the top of the contents to any
%% \cvsection title in the "main" bar.

\cvsection[page1sidebar]{Experience}

%%%%%%%%%%%%%%%%%%%%%%%%%%
% Experience 1
% Lead Software Engineer
%%%%%%%%%%%%%%%%%%%%%%%%%%

\cvexp
{Lead Software Engineer}
{Nagra Kudelski India}
{July 2019 -- Present}
{Bangalore, India}



\begin{itemize}

    
    \item 
    Developed \textbf{cloud-based microservices} using \textbf{Java 8} and \textbf{Spring Boot}.
    
    \item 
    Collaborated with technical architects to create \textbf{high-level design} and \textbf{low-level design} using UML diagrams.
    
    \item 
    \textbf{Mentored} junior members of the team.
    
    \item
    Implemented \textbf{docker} based \textbf{containerized integration tests} and \textbf{unit tests}. Maintaining the \textbf{code coverage} above 65\%.
    
    \item
    Actively did \textbf{pair-programming} for \textbf{code reviews} and technical challenges.
    
\end{itemize}

\divider


%%%%%%%%%%%%%%%%%%%%%%%%%%
% Experience 1
% Senior Software Engineer
%%%%%%%%%%%%%%%%%%%%%%%%%%

\cvexp
{Senior Software Engineer}
{Nagra Kudelski India}
{October 2014 -- June 2019}
{Bangalore, India}

\begin{itemize}

    \item 
    Designed and implemented web services, domain and database components using \textbf{Service-Oriented Architecture}.
    
    \item 
    Worked by \textbf{Agile Scrum} framework. Responsibilities included interaction with business team in story grooming, estimations, defining acceptance criteria, definition of done.  
    
    \item 
    \textbf{Bug-fix} and tracked issues using \textbf{JIRA}, version control in \textbf{GitLab}.
    
\end{itemize}

\divider

%%%%%%%%%%%%%%%%%%%%%%%%%%
% Experience 3
% Software Engineer
%%%%%%%%%%%%%%%%%%%%%%%%%%


\cvexp
{Software Engineer}
{Cognizant Technologies Solutions}
{July 2013 -- October 2014}
{Coimbatore, India}

\begin{itemize}

    \item 
    Developed, maintained software using \textbf{Java 6} and \textbf{MySQL}.
    
    \item 
    Tracked bugs \& defects via HP ALM tool.
    
\end{itemize}

% \divider


\cvsection{Projects}


\cvprojects
{Project 1 - Business Enabling Interface}
{Nagra Kudelski India}
{6em}
{September 2019 -- Present}
% {Bangalore, India}


\begin{itemize}

    \item 
    BEI provides interfaces and administration to provide integration of online services to customers to implement value added business solutions. 
    
    \item
    Transformed monolithic app to micro services using \textbf{Spring Boot} via  \textbf{12-factor app} methodology \textbf{Centralized Configuration \& Logging}. Revamped the logging architecture by following \textbf{Chain of Responsibility} design pattern.
    
    \item
    Developed \textbf{load balancing} module using \textbf{Netflix Ribbon}. Used \textbf{    Weighted Response Time Rule} for improving time efficiency. Implemented analytical data for publishing alive nodes, total requests, connection history etc.
    
%     \item 
%     Routing - Used Netflix OSS Zuul for upstream/downstream proxy. Proxy - HA-Proxy (TCP/HTTP Load Balancer/Reverse Proxy), Squid 
%     (Caching proxy).
    
%     \item 
%     Increased fault tolerance between web components by implementing Circuit-Breaker Pattern using Hystrix. Introduced multiple Hystrix Circuits to increase fault tolerance and reliability of system with proper fallback mechanism during downtime.
    
% 	\item 
% 	Load Balancing - used Netflix Ribbon library. Used WeightedResponseTime rule for improving time efficiency. Implemented PingService for Alive check. Implemented LoadBalancerStats for publishing Active-servers, total requests, connection history (first, successive, last).
	
    % \item 
    % Health monitor - Implemented RESTFul API similar to Spring Boot actuators to monitor - DB Datasources, Thread pools(active, peak threads, queue size, rejected pool).
    
    \item \textbf{\textcolor{accent}{Tech used :}} 
    \textcolor{accent}{
    Spring Boot, Spring Security, Spring data JPA, REST web-services, Microservice architecture,  Docker, Kubernetes, PostgreSQL, RabbitMQ, AWS S3, AWS EC2, AWS EKS, AWS Cloud Watch, AWS IAM}
    
\end{itemize}


\divider


\cvprojects
{Project 2 - Skidata Sales Platform}
{Nagra Kudelski India}
{6em}
{April 2017 -- September 2019}
% {Bangalore, India}

\begin{itemize}

    % \item 
    % SSP provides the Sales platform to SKIDATA with cloud solution stack. SSP serves as a 24x7 SaaS realized as a platform for web-based consumer portal. Order service to place the order, Payment service for payments, Catalog for elastic search.
    
    \item
    Increased fault tolerance between components by implementing \textbf{Circuit-Breaker Pattern} using \textbf{Hystrix}. Increased system reliability with proper fallback mechanism during downtime.
    
    \item
    Improved robustness by implementing health monitor RESTFul API using \textbf{Spring Boot Actuators} to monitor data-sources, thread pools, active threads, peak threads, rejected threads.
    
    \item \textbf{\textcolor{accent}{Tech used : }} 
    \textcolor{accent}{
    RabbitMQ, Spring boot, Spring data JPA, REST web-services, Microservice architecture, Docker, Rancher, PostgreSQL}

\end{itemize}

% \divider


% \cvheading{Test Automation}

% \begin{itemize}
%     \item Docker based test containers for deploying .ear/.jar/.war files. Integrating shell scripts with Dockerfile
%     \item Setting up test data using SQL queries, utility fixture. Wildfly properties configuration using Shell Script.
%     \item JUnit Test for backend SOAP/REST API - for API testing. Test coverage to sonarQube - using maven-jacoco-plugin
% \end{itemize}

% GENERAL CONTENT
% \begin{itemize}
%     \item Developing software using Java 8
%     \item Build API using SOAP, REST web services
%     \item Docker based test containers for deploying .ear/.jar/.war files
%     \item Wildfly application server properties configuration using CLI and Shell Script
%     \item Integrating shell scripts with Dockerfile
%     \item JUnit Test for backend SOAP/REST API - for API testing
%     \item API documentation - AsciiDoc (generate HTML, PDF format)
%     \item Swagger - for RESTFul API doc
% \end{itemize}

% \divider

%%%%%%%%%%%%%
%Experience 2
%%%%%%%%%%%%%

% \cvevent
% {Programmer Analyst}
% {Cognizant Technologies Solutions Pvt. Ltd}
% {July 2013 -- October 2014}
% {Coimbatore, India}
% {Project - CIGNA (Heathcare)}
% \begin{itemize}
%     \item Created scripts using parameterization, Descriptive programming, dynamic web objects.
%     \item Hybrid 4-Tier Automation Framework for executing the VB Scripts
%     \item Used various components of Automation Framework like, Driver, Utility Functions, Business Components, Object Repository, Reporting Utilities and Execution Results.
% \end{itemize}

%%%%%%%%%%%%%%%%%%%%%%%%%%%%%%% Achievements %%%%%%%%%%%%%%%%%%%%%%%%%%%%%%%

% \cvsection{Achievements}
% \smallskip
%     \begin{itemize}
%     \item OCJP
%     \smallskip
% \end{itemize}


\clearpage

% \cvsection[page2sidebar]{Education}

% \nocite{*}

% \printbibliography[heading=pubtype,title={\printinfo{\faBook}{Books}},type=book]

% \divider

% \printbibliography[heading=pubtype,title={\printinfo{\faFileTextO}{Journal Articles}}, type=article]

% \divider

% \printbibliography[heading=pubtype,title={\printinfo{\faGroup}{Conference Proceedings}},type=inproceedings]

% %% If the NEXT page doesn't start with a \cvsection but you'd
% %% still like to add a sidebar, then use this command on THIS
% %% page to add it. The optional argument lets you pull up the
% %% sidebar a bit so that it looks aligned with the top of the
% %% main column.
% % \addnextpagesidebar[-1ex]{page3sidebar}


\end{document}
